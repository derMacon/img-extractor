\documentclass[pdf]{beamer}

\usetheme[
	subsectionpage=progressbar,
	progressbar=frametitle,
	sectionpage=none,
	block=fill
]{metropolis}

\title{Image Extractor}
\subtitle{Optimze your Anki Workflow}
\author{
    Silas Hoffmann
}

\begin{document}

\maketitle

\begin{frame}{Main Working Principle}
    My typical workflow when creating new cards in Anki typically consists of having a window with the script of the lecture open, and another with the actual anki app. 
    I then make a screenshot of the currently displayed page of the script and create a set of questions which the lecturer might ask in an exam. The questions belong
    on the front of the card, while the screenshots come on the back. This results in a lot of switching between the active windows, and is quite time consuming and frustrating.
    In order to optimize this workflow I wrote this program which can be configured to use hotkeys to turn pages and the pdf window does not need to be selected by the user.
    It is possible to just stay in the Anki window, navigate via shortcuts and get the currently displayed page copied directly to the system clipboard for the user to paste.

    \subsection{typical workflow}
    \begin{itemize}
        \item turn to next page via shortcut
        \item think of a question for the front of the card
        \item paste the currently displayed of the pdf to back of the card
        \item turn to next page via shortcut
    \end{itemize}
\end{frame}

\begin{frame}{Navigation}
    \subsection{Shortcuts}
    \begin{itemize}
        \item turn to next page
        \item turn to previous page
    \end{itemize}

    \subsection{Browser Interface}
    \begin{itemize}
        \item Button turn to next page
        \item Button turn to previous page
        \item Button go to specific page
        \item Button to open previously opened pdf at last position
        \item Button to open a new pdf 
    \end{itemize}

\end{frame}



\end{document}